\chapter{Introduction}

AVDTA is a research software for studying DTA and mesoscopic simulation models of autonomous vehicle technologies.
%
AVDTA extends the typical kinematic wave theory mesoscopic models of simulation-based DTA by adding a number of revised models based on CAV technologies. These include the following:
\begin{itemize}
	\item Multiclass cell transmission model where the fundamental diagram changes in space and time\cite{levin2016multiclass}
	\item Dynamic lane reversal\cite{levin2016cell, duell2016system}
	\item Conflict region model of reservation-based intersection control\cite{levin2015intersection, levin2015optimizing}
	\item Backpressure and $P_0$ policies for reservation-based intersection control
	\item Dynamic transit lanes
\end{itemize}

This documentation is divided into two parts. Part \ref{part:ui} explains the user interface and input data formats, and is intended for users who wish to use the existing software. The user interface contains two parts: a control interface that can modify all network, transit, and demand data and run DTA, and an editor that can visualize the network and modify individual nodes and links. For software development, Part \ref{part:api} discusses the code structure and gives some examples of working with the code. Most of the code, including all classes relevant to traffic flow and simulation, also has Javadocs.

\section{Contributors}
Most software development was by Michael W. Levin. Sudesh Agrawal contributed to calibration and testing of the conflict region model and cell transmission model. Josiah P. Hanna and Guni Sharon contributed to the microtolling package. 

\section{Distribution}
 AVDTA is not available for commercial use, and may not be used or distributed without the permission of the authors.



%\section{Support}
%Documentation for the API is available in the \texttt{Documentation/javadoc} folder. The author may be contacted at \href{mailto:michaellevin@utexas.edu}{\texttt{michaellevin@utexas.edu}}.