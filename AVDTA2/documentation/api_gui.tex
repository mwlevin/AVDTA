\chapter{Package \texttt{avdta.gui}}

The GUI has less Javadocs available than the remaining packages, which are more related to traffic network modeling. However, Javadoc would be less helpful for the \texttt{avdta.gui} package as much of the code exists in the constructors. This chapter will give an overview of the classes, and how to modify them to account for new nodes and links. There are two parts to the GUI: first, the \texttt{avdta.gui.GUI} and subclasses define the DTA functional GUI. Most of the code is contained within the \texttt{avdta.gui.panels} package. The Editor is contained within the \texttt{avdta.gui.editor} package. 
%
The GUI packages are organized as follows:
\begin{list}{$\circ$}{}
	\item \texttt{avdta.gui}
	
	\begin{list}{$\circ$}{}
		\item \texttt{avdta.gui.editor}
		
		\begin{list}{$\circ$}{}
			\item \texttt{avdta.gui.editor.visual}
			
			\begin{list}{$\circ$}{}
				\item \texttt{avdta.gui.visual.rules}
				
				\begin{list}{$\circ$}{}
					\item \texttt{avdta.gui.editor.visual.rules.data}
					\item \texttt{avdta.gui.editor.visual.rules.editor}
				\end{list}
			\end{list}
		\end{list}
		
		\item \texttt{avdta.gui.panels}
		
		\begin{list}{$\circ$}{}
			\item \texttt{avdta.gui.panels.analysis}
			\item \texttt{avdta.gui.panels.demand}
			\item \texttt{avdta.gui.panels.dta}
			\item \texttt{avdta.gui.panels.network}
			\item \texttt{avdta.gui.panels.transit}
		\end{list}
		
		\item \texttt{avdta.gui.util}
	\end{list}
	
\end{list}

\section{Package \texttt{avdta.gui.panels}}













\section{Package \texttt{avdta.gui.util}}
\label{api:util}



\section{Package \texttt{avdta.gui.editor}}