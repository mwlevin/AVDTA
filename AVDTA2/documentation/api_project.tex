\chapter{Package \texttt{avdta.project}}
\label{api:project}

\texttt{Project}s provide an API for reading and writing data, given that data appears in a specific format. AVDTA is based around using \texttt{Project}s to retrieve data due to the simplicity. {Project}s contain data access interfaces such as 
\begin{algorithmic}[1]
\State{\texttt{Project.getNodesFile()}}
\State{\texttt{Project.getLinksFile()}}
\end{algorithmic}
with similar methods for all relevant data files. 

\section{Organization}
\texttt{Project}s are organized according to various data requirements. Initial levels of abstraction (\texttt{Project}, \texttt{TransitProject}, \texttt{DemandProject}) are abstract classes and not meant to be instantiated. \texttt{DTAProject}, \texttt{FourStepProject}, and \texttt{SAVProject} are subclasses that refer to types of projects (DTA, four-step planning model, and SAVs, respectively). The entire \texttt{avdta.project} package is shown below, with abstract classes italicized.
%
\begin{list}{$\circ$}{}
	\item \textit{\texttt{avdta.project.Project}}
	\begin{list}{$\circ$}{}
		\item \textit{\texttt{avdta.project.TransitProject}}
		
		This class extends \texttt{Project} to add transit-related data.
		
		\begin{list}{$\circ$}{}
			\item \textit{\texttt{avdta.project.DemandProject}}
			
			This class extends \texttt{TransitProject} to add demand-related data.
			
			\begin{list}{$\circ$}{}
				\item \texttt{avdta.project.DTAProject}
				
				\begin{list}{$\circ$}{}
					\item \texttt{avdta.project.FourStepProject}
				\end{list}
			
				\item \texttt{avdta.project.SAVProject}
			\end{list}
		\end{list}
	\end{list}
\end{list}
%
\texttt{Project} represents a project with standard network data (nodes, links, intersection controls). \texttt{TransitProject} adds transit data (buses), and \texttt{DemandProject} adds traveler demand data, which may be used to create travelers or personal vehicles.

\section{Working with \texttt{Project}s}

This section discusses how to create and open \texttt{Project}s, and presents some code examples.

\subsection{Creating a new project}
\texttt{Project}s follow a very specific file format, and it is therefore recommended not to attempt to create a \texttt{Project} folder by hand. Instead, AVDTA provides a method to create new projects, \texttt{Project.createProject(String, File)}. For instance,
\begin{algorithmic}[1]
\State{\texttt{DTAProject project = new DTAProject();}}
\State{\texttt{project.createProject("Sioux Falls", \linebreak new File("projects/SiouxFalls"));}}
\end{algorithmic}
will create all the files used by a \texttt{DTAProject}. All files will be empty except for the header line, and data should be copied in or imported from other sources. 

\subsection{Importing data}
There are several methods to import data. \texttt{Project.cloneFromProject(Project)} will copy appropriate data from the parameter \texttt{Project}. The data copied will depend on the types of projects; for instance, a \texttt{DemandProject} would attempt to copy demand data if the parameter is also a \texttt{DemandProject}. Data can also be imported from VISTA\cite{ziliaskopoulos2000internet}, but the methods to do so are separated into other packages based on data type. See
%
\begin{itemize}
	
\item \texttt{avdta.network.ImportFromVISTA}

This imports network and traffic flow-related data files (see Chapter \ref{ch:network}).

\item \texttt{avdta.transit.TransitImportFromVISTA}

This imports transit-related data files (see Chapter \ref{ch:transit}).

\item \texttt{avdta.demand.DemandImportFromVISTA}

This imports demand-related data files (see Chapter \ref{ch:demand}).

\item \texttt{avdta.dta.DTAImportFromVISTA}

This uses the \texttt{vehicle\_path} and \texttt{vehicle\_path\_time} tables from VISTA to create an assignment within AVDTA.
\end{itemize}
These classes require specifying the input files (which are VISTA database tables) and the output \texttt{Project}.

\subsection{Opening a project}

\texttt{Project}s also include a constructor \texttt{Project.Project(File)}, which constructs the \texttt{Project} using the specified project folder. This should be the same file used when the project was created via \texttt{Project.createProject(String, File)}. The usual purpose of opening a project is to access its \texttt{Simulator} (see Cha \ref{api:network}), which is accomplished via
\begin{algorithmic}[1]
\State{\texttt{DTAProject project = \linebreak new DTAProject(new File("projects/SiouxFalls"));}}
\State{\texttt{DTASimulator sim = project.getSimulator();}}
\end{algorithmic}
The \texttt{Simulator} can also be loaded manually via \texttt{Project.loadSimulator()}. However, calling \texttt{Project.getSimulator()} will call \texttt{Project.loadSimulator()} if necessary.